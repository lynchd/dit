\documentclass[10pt,a4paper]{article}
\usepackage[T1]{fontenc}
\usepackage[scaled]{helvet}
\usepackage{cite}
\usepackage{url}
\usepackage{graphicx}
\usepackage{listings}
\usepackage{float}
\usepackage{amsmath}
\usepackage{listings}
\usepackage{color}
 
\definecolor{dkgreen}{rgb}{0,0.6,0}
\definecolor{gray}{rgb}{0.5,0.5,0.5}
\definecolor{mauve}{rgb}{0.58,0,0.82}
\lstset{ %
  language=Octave,                % the language of the code
  basicstyle=\footnotesize,           % the size of the fonts that are used for the code
  numbers=left,                   % where to put the line-numbers
  numberstyle=\tiny\color{gray},  % the style that is used for the line-numbers
  stepnumber=1,                   % the step between two line-numbers. If it's 1, each line 
                                  % will be numbered
  numbersep=5pt,                  % how far the line-numbers are from the code
  backgroundcolor=\color{white},      % choose the background color. You must add \usepackage{color}
  showspaces=false,               % show spaces adding particular underscores
  showstringspaces=true,         % underline spaces within strings
  showtabs=false,                 % show tabs within strings adding particular underscores
  frame=none,                   % adds a frame around the code
  rulecolor=\color{black},        % if not set, the frame-color may be changed on line-breaks within not-black text (e.g. commens (green here))
  tabsize=2,                      % sets default tabsize to 2 spaces
  breaklines=true,                % sets automatic line breaking
  breakatwhitespace=false,        % sets if automatic breaks should only happen at whitespace
  keywordstyle=\color{blue},          % keyword style
  commentstyle=\color{dkgreen},       % comment style
  stringstyle=\color{mauve},         % string literal style
  escapeinside={\%*}{*)},            % if you want to add LaTeX within your code
  morekeywords={*,...}               % if you want to add more keywords to the set
}
\usepackage{amssymb}
\usepackage{fancyhdr}
\usepackage{lastpage}
\floatstyle{boxed} 
\restylefloat{figure}
\renewcommand*\familydefault{\sfdefault}
\title{Synchronization}
\author{David Lynch - david.lynch@raglansoftware.com }
\begin{document}
\maketitle
\begin{abstract}
To date we have talked about concurrently running threads of execution, but have not addressed any of the fundamental protections that must be afforded to state that is shared between different threads. We have already alluded to a number of problems with data consistency and determinism that are surfaced by multi-threaded execution. This article examines these issues in detail and discusses some protections that operating systems provided to mitigate against these issues.
\end{abstract}
\section{The Critical Section Problem}
\subsection{Petersons Solution}
\section{Concurrency Primitives}
\subsection{Mutex}
\subsection{Semaphores}
\subsection{Monitors}
\section{Deadlock, Livelock and Starvation}
\subsection{The Bounded Buffer Problem}
\subsection{The Dining Philosophers Problem}
\section{Transactions}

\bibliography{../biblio/techfundamentals.bib}{}
\bibliographystyle{plain}
\begin{center}
{\small \copyright  David Lynch 2012. Do not reproduce without written permission.}
\end{center}
\end{document}
