\documentclass[10pt,a4paper]{article}
\usepackage[T1]{fontenc}
\usepackage[scaled]{helvet}
\usepackage{cite}
\usepackage{url}
\usepackage{graphicx}
\usepackage{listings}
\usepackage{float}
\usepackage{amsmath}
\usepackage{listings}
\usepackage{color}
 
\definecolor{dkgreen}{rgb}{0,0.6,0}
\definecolor{gray}{rgb}{0.5,0.5,0.5}
\definecolor{mauve}{rgb}{0.58,0,0.82}
\lstset{ %
  language=Octave,                % the language of the code
  basicstyle=\footnotesize,           % the size of the fonts that are used for the code
  numbers=left,                   % where to put the line-numbers
  numberstyle=\tiny\color{gray},  % the style that is used for the line-numbers
  stepnumber=1,                   % the step between two line-numbers. If it's 1, each line 
                                  % will be numbered
  numbersep=5pt,                  % how far the line-numbers are from the code
  backgroundcolor=\color{white},      % choose the background color. You must add \usepackage{color}
  showspaces=false,               % show spaces adding particular underscores
  showstringspaces=true,         % underline spaces within strings
  showtabs=false,                 % show tabs within strings adding particular underscores
  frame=none,                   % adds a frame around the code
  rulecolor=\color{black},        % if not set, the frame-color may be changed on line-breaks within not-black text (e.g. commens (green here))
  tabsize=4,                      % sets default tabsize to 2 spaces
  breaklines=true,                % sets automatic line breaking
  breakatwhitespace=false,        % sets if automatic breaks should only happen at whitespace
  keywordstyle=\color{blue},          % keyword style
  commentstyle=\color{dkgreen},       % comment style
  stringstyle=\color{mauve},         % string literal style
  escapeinside={\%*}{*)},            % if you want to add LaTeX within your code
  morekeywords={*,...}               % if you want to add more keywords to the set
}
\usepackage{amssymb}
\usepackage{fancyhdr}
\usepackage{lastpage}
\floatstyle{boxed} 
\restylefloat{figure}
\renewcommand*\familydefault{\sfdefault}
\title{Programming Languages and Paradigms}
\author{David Lynch - david.lynch@raglansoftware.com }
\begin{document}
\maketitle
\begin{abstract}
NEEDS ABSTRACT
\end{abstract}
\section{Programming Languages}
Solving programming problems involves choosing the right tools and concepts for the challenge at hand. There is no consensus for choosing the {\bf correct} programming language or paradigm. That is not to say we cannot evaluate programming languages and paradigms across some criteria. 
\subsection{Readability}
The ease at which programs can be read and understood is of vital importance. The majority of the lifecycle of any significant application is maintenance and addition of new features. You will spend a large portion of your time as a programmer in maintenance, and therefore reading either your own code or the code of others. For a specific problem domain, expressions of a problem in a certain programming language may be unnatural and convoluted. This adds to impedance to the ability of a code-base to change. There are a number of things that contribute to the overall readability of a programming language. 
\subsubsection{Simplicity}
Overall simplicity of a programming language refers not only to the complexity of its basic components, but also to the number of basic components and the multiplicity of its features. 
\subsubsection{Orthagonality}
Orthagonality states that a relatively small set of primitive constructs can be combined in a small number of ways to build the control structures of the language. For example, having the primitive types INTEGER, FLOAT, DOUBLE and CHARACTER combined with constructs ARRAY and POINTER allows combinations that support complex data structures such as the ones we have studied so far. Lack of orthagonality leads to exceptions of the rules of the language. Fewer exceptions leads to a higher degree of regularity in the language itself. Simplicity in the language, overall, is in part the result of a combination of a relatively small number of primitive constructs and limited orthogonality. Some argue that functional languages strike a good balance between orthagonality and simplicity when contrasted with imperative languages such as C++.
\subsubsection{Control Statements}
Examples of control statements are {\it if, else, while, for and continue}. Use of control statements aid readability by facilitating writing programs that can be read from top to bottom. Indiscriminate use of the goto statement is s good example of how lack of complexity in constructs can contribute to lack of readability in programs. Most, if not all, modern programming languages now add sufficent control statements to elminate the use of goto completely. Languages such as Ruby even provide means of improving upon the readability of traditionaly C style constructs such as the for loop. See the Ruby {\it each} array operation for an example. 
\subsubsection{Data-Types and Structures}

\subsubsection{Syntax}

\subsection{Writablity}
\subsubsection{Simplicity}
\subsubsection{Orthagonality}
\subsubsection{Abstraction}
\subsubsection{Expressivity}


\subsection{Reliability}
\subsubsection{Exception Handling}
\subsubsection{Aliasing}

\subsection{Cost}

\section{Evaluation}
\subsection{Imperative}
\subsection{Declarative}
\subsection{Procedural}
\subsection{Object Oriented}
\subsection{Functional & Logical}

\subsection{

{\small \copyright  David Lynch 2012. Do not reproduce without written permission.}
\end{center}
\end{document}
